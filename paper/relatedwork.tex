\section{Related Work}
\label{sec:related-work}

The first encounter-registration system has been developed by a company called Sirtrack 
over a decade ago~\cite{Prangle2006NewRadiocolars}. The tags
were placed on collars that were attached to wild mammals. The tags are fairly heavy, weighing 45--450g, depending on the
size of battery. Tags transmit every 1.5~s, and are able to record encounters of 15~s or longer. The system is
commercial has has been used extensively, for example, to study the possibility of disease transmission between cattle 
and wild badgers~\cite{Bohm2009WildlifeLivestock}.

The next major advance was a system called {\em Enounternet}~\cite{Menhill2012NovelTelemetry,Rutz2012AutomatedMapping}, 
which has been available commercially for a few years
but is no longer available. The key innovation in Encounternet was weight: tags used a tiny printed-circuit board and
could be powered by miniature batteries, allowing tags weighing 1.3g and up to be manufactured. Clearly, the small
batteries restrict the life-span of tags and increase the importance of effective protocols, especially with respect
to low-power operation. The small size of encounternet tags has enabled the study of interaction of small species, 
including small birds~\cite{Levin2015Performance} and freshwater fish~\cite{Tentelier2016FishNetwork}, to cite just a few.
Another innovation of the Encounternet system has been the recording of RSSI with every reception report, allowing
researchers to estimate the distance of each encounter. 

Interestingly, none of the papers that describe these systems and 
ecological research carried out with them gives any details on the
MAC-layer protocol that was used, nor on how the receiver is duty cycled. In particular, it appears that many of these 
protocols are not particularly efficient. For example, switching an Encounternet tag with a particular battery from
transmit-only mode (intended to record its proximity to base stations) to encounter-registration mode reduces the lifespan
of the tag from 7.5 days to less than a day, indicating that the receiver is active a significant fraction of the time.

On the positive side, most of these papers do include validation and calibration studies intended to quantify the distances
at which encounters are recorded. We report below on a similar study carried out by us.

More recently, prototype Encounternet-like tags have been developed and tested on
bats~\cite{Ripperger2016ProximitySensing,dressler2016bats}. These tags include a wake-up receiver, which
enables nano-power listening without maintenance of inter-tag synchronized clocks. These tags  
do not appear to have been commercialized or widely-used otherwise.

SHEN TONG SHOULD ADD PROTOCOL-RELATED WORK.


