\section{Related Work}
\label{sec:related-work}

\subsection{Encounter-registration system}
The first encounter-registration system was developed by a company called Sirtrack 
over a decade ago~\cite{Prangle2006NewRadiocolars}. The tags
are placed on collars attached to wild mammals. 
They are fairly heavy, weighing 45--450g, depending on the
size of battery. % Tags transmit every 1.5~s, and are able to record encounters of 15~s or longer. 
The system has been commercially used extensively. For example, it is used to study 
the possibility of disease transmission between cattle 
and wild badgers~\cite{Bohm2009WildlifeLivestock}.

The next generalization was a system called {\em Enounternet}~\cite{Menhill2012NovelTelemetry,Rutz2012AutomatedMapping}, 
which has been available commercially for a few years
but is no longer available. The key innovation of Encounternet was tags' weight: 
tags used a tiny printed-circuit board and
could be powered by miniature batteries, allowing tags weighing 1.3g and up to be manufactured. 
Clearly, the small batteries restrict the life-span of tags and increase the importance of effective protocols, 
especially with respect to low-power operation. 
The small size of Encounternet tags has enabled the study of interaction of small species, 
including small birds~\cite{Levin2015Performance} and freshwater fish~\cite{Tentelier2016FishNetwork}.
Another innovation of the Encounternet system is the recording of RSSI with every reception report, allowing
researchers to estimate the distance of each encounter. 

More recently, prototype Encounternet-like tags have been developed and tested on
bats~\cite{Ripperger2016ProximitySensing,dressler2016bats}. These tags include a wake-up receiver, which
enables nano-power listening without maintenance of inter-tag synchronized clocks. These tags  
do not appear to have been commercialized or widely-used otherwise.




\subsection{Protocol for encounter registration}


Unfortunately, none of the papers that describe these systems and 
ecological research give any details on the
MAC-layer protocol that is used, nor on how the receiver is duty cycled. 
Existing methods to encounter registration problem are mainly based on fixed transmitting 
probability~\cite{Menhill2012NovelTelemetry,Rutz2012AutomatedMapping}
(i.e., agents transmit a beacon with a fixed probability $p$ and listen with $1-p$). 
In particular, it appears that many of these 
protocols are not particularly efficient. For example, switching 
an Encounternet tag with a particular battery from
transmit-only mode 
to encounter-registration mode reduces the lifespan
of the tag from 7.5 days to less than a day, indicating that the 
receiver is active a significant fraction of the time.

Although to the best of our knowledge, there has been no analyses of the protocols 
for encounter registration problem,
several similar problems are well-studied in the wireless network literature, 
such as  
minimum dominating set problem~\cite{Scheideler2008An,Yu2013Review},
neighbor discovery problem~\cite{Bakht2012Searchlight, Sun2014Hello,Chen2015On}
and information exchange problem~\cite{Capetanakis1979Tree,Daum2013Maximal,Yu2017Uniform}.

Minimum dominating set problem~\cite{Scheideler2008An,Yu2013Review} is studied to 
deal with interference challenges for the wireless multi-hop networks.
This problem focus on the communication models, such as unit disk graph model~\cite{Lebhar2009Unit} (UDG),
graph-based model~\cite{De2007A} and Signal to Interference plus Noise Ratio model~\cite{Lee2007Signal} (SINR).
Neighbor discovery problem is well studied in the wireless sensor 
networks. Many algorithms~\cite{Bakht2012Searchlight, Sun2014Hello,Chen2015On} 
are designed for two nodes to discover each other and 
are applied directly to the multi-node scenario.
Information exchange problem~\cite{Capetanakis1979Tree,Daum2013Maximal,Yu2017Uniform} is 
studied on the information propagation in a single-hop network. 
In the network, there are active nodes with packets to transmit 
and inactive nodes waiting to receive.
A common solution to these problems is to control the transmission probability of the node in the 
network, making it optimal to transmit successfully.
Some deterministic methods, such as quorum system~\cite{Peleg1995The} and prime number, are also 
adopted to help synchronization in the protocol design.
The major difference between the problems above and the encounter registration problem
in this paper is the mobility of the wild animals. It makes the problem 
complicated that whether an encounter is happening at the moment is unknown.
% Besides the encounter behavior varies from individual to individual. Some animals are eager to 
% keep company with peers while others stay alone most of the time.
% Since animals are attached with energy-restrained
% tags, the energy-efficiency should also be taken into consideration.
% All these factors make the problem challenging.
  

