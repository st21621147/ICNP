\section{Analysis of the {\pName} Protocol}
\label{sectionanalysis}

In this section, we prove that with a clique of $k$ agents,  
{\pName} protocol guarantees each agent can record all
the peers in $O(k)$ slots with high probability.
Formally, this conclusion is derived from the following theorem.
\begin{theorem}
    \label{bound}
    Consider an encounter process of $k$ agents :
    \begin{itemize} 
    \item[(1)] An agent in detecting stage  will switch to the connecting stage in $O(f(\theta))$
    slots with high probability.
    \item[(2)] When all the agents are in the connecting stage, each agent can 
    successfully transmit once and turn to \emph{quiet} state in $O(k)$ slots with high probability.
    \end{itemize}
\end{theorem}

We prove the first conclusion in Theorem~\ref{bound} in Section~\ref{boundPC}, and the second
conclusion in Section~\ref{boundSW}. 
Finally, We explain how the {\pName} protocol deal with the dynamic movements
of agents and record a duration encounter process in Section~\ref{}.

\subsection{Time bound for detecting stage}
\label{boundSW}

When encouter happens, an agent in detecting stage will switch to connecting stage very soon. 
We derive this conclusion form the following lemma.
\begin{lemma}
    An agent in detecting stage  will switch to the connecting stage in 
    $\frac{\ln\eta}{\ln(\frac{1}{2}+\frac{1}{2^k})} \lceil \frac{9}{4\theta^{2}} \rceil$
    slots with high probability.

\end{lemma}
\begin{IEEEproof}
    Consider two agents encounter at a time and one turn to the connecting stage first.
    Then the other will switch to the connecting stage very soon since the prior one in connecting
    stage increases its transmission probability (because there is no duty cycle in the 
    connecting stage) to let other peers detect it. 

    Now we consider $k$ agents in the detecting stage at
    the beginning. 
    For any pair of agents ($u_i$, $u_j$),
    we can find an ordered pair ($r_i$, $r_j$) from the constructed RDS
    such that $r_i - r_j \equiv \delta_{ij}$ (mode $T_0$), where $\delta_{ij}$ is the time drift.
    This indicates that any two peers can turn on their radios in the same 
    slot at least once during every period of $T_0$. 
    For a specific agent, the probability it detects a peer in a rounds of $T_0$ is at least
    $Pr \geq 1 - (\frac{1}{2} +\frac{1}{2^k})$.
    Hence it holds with high probability that an agent detects peer(s) in 
    $\frac{\ln\eta}{\ln(\frac{1}{2}+\frac{1}{2^k})} \lceil \frac{9}{4\theta^{2}} \rceil$ slots,
    where $\eta$ is small enough.    
\end{IEEEproof}


\subsection{Time bound for connecting stage}
\label{boundPC}
Consider the connecting stage.
We analyze the upper bound of slots
for all the agents in the clique to transmit successfully.
Note that, an agent transmitting successfully will be recorded by 
all the other peers in the clique, and then it switches to the \emph{quiet} state. 
Denote $S_t$ as the set of agents
which have not switched to the \emph{quiet} state in time slot $t$, and 
$|S_t|$ is the cardinality. 
Thus the upper bound of the protocol is the maximum time slots
for a clique to turn to $|S_t| = 0$.

In the following, we first show the
upper bound of slots it takes from the beginning to $|S_t| \leq \log k$ in Lemma~\ref{fast}.
Then we show the upper bound it takes to $|S_t| = 1$ in Lemma~\ref{slow}.
Finally we present that it takes for all agents to successfully transmit~(i.e, $|S_t| = 0$.) in $O(k)$ in Lemma~\ref{one}. 

\begin{lemma}
    \label{fast}
    At time $T_2=T_1 + \gamma_1\cdot k  =O(k)$ 
    it holds with high probability that $|S_{T_2}| \leq \log k$. 
\end{lemma}

To prove this lemma, we need to introduce and prove some small lemmas at first.
We review two useful lemmas as follows.
\begin{lemma}
    \label{sumP}
    Consider a set of $l$ agents, $u_1, u_2,\dots, u_l$.
    For an agent $u_i$, it transmits with probability 
    $0 < \omega(u_i) < \frac{1}{2}$. Let $p_0$ denote the 
    probability that the channel is idle in a time slot; and $p_1$ denote 
    the probability that there is exactly one transmission in a time slot.
    Then $p_0\cdot {\sum}_{i=1}^{l}\omega(u_i)\leq p_1 \leq
    2\cdot p_0\cdot {\sum}_{i=1}^{l}\omega(u_i)$     
\end{lemma}
\begin{lemma}
    \label{mulP}
    With $a_i \in [0, \frac{1}{2}]$ for $i = 0,1,\dots,$ it holds that
    \begin{equation}
        \label{mul}
        4^{-{\sum}_{i}a_i}\leq {\prod}_{i}(1-a_i)\leq e^{-{\sum}_{i}a_i}. 
    \end{equation}
\end{lemma}
The proof of Lemma~\ref{sumP} can be referred to~\cite{Richa2010A}
and the proof of Lemma~\ref{mulP} can be referred to~\cite{Daum2013Maximal}.
Based on these two lemmas, we get the following conclusion.

\begin{lemma}
    \label{safe}
    For a time slot $T>0$ with $|S_t| \ge \log k$, if 
    there exist constants $\alpha_1, \alpha_2 \ge 1$ such that
    $\alpha_1 \leq \sum_{u\in S_t}{\omega}_t(u) \leq \alpha_2$, then with constant probability 
    there is one active node switching to the \emph{quiet} state
    in each time slot. 
\end{lemma}
\begin{IEEEproof}
    In each time slot, the channel is idle with probability at least 
    $4^{-2\alpha}$, and there is exactly one transmission on the channel
    with probability at least $\alpha\cdot 4^{-2\alpha}$. By the Chernoff
    bound, it holds that with constant probability (given $\alpha_1$ and $\alpha_2$),
    there are one agent switch to the quiet state in each slot.
\end{IEEEproof}

Next we show that after all the agents turn to the connecting stage, 
the summation of the total agents' transmission probabilities will go
between $\alpha_1$ and $\alpha_2$ soon, where $\alpha_1$ and $\alpha_2$
are constants defined in Lemma~\ref{safe}.
\begin{lemma}
    \label{lemma7}
    For a time slot with ${\sum}_{u\in S_t}\omega_t(u)=\alpha$, it holds that
    $Pr[{\sum}_{u\in S_t}\omega_{t+1}(u) \leq \alpha\cdot\frac{3}{4}] 
    \geq \frac{7}{8}$
    for large enough $\alpha$.
\end{lemma}
\begin{IEEEproof}
    The probability that there are more than one agents transmit in time slot 
    $t$ is at least $1-exp\{-\alpha\}$ according to Equation~(\ref{mul}).
    All the agents will halve their transmission probabilities if the channek is 
    not idle in slot $t$. Denote X as the random variable that indicates the value of 
    ${\sum}_{u\in S_{t+1}}\omega_{t+1}(u)$. We get,
    \begin{equation*}
        Pr[X ={\sum}_{u\in S_t}\frac{\omega_t(u)}{2}] 
        \geq 1 - exp\{-\alpha\}
    \end{equation*}
    which is at least $7/8$ when $\alpha$ is large enough. Hence,
    \begin{equation*}
        X \leq \frac{7}{8}\cdot\frac{1}{2}\cdot\alpha + 
        \frac{1}{8}\cdot 2\alpha 
        < \frac{3}{4}\cdot\alpha.
    \end{equation*}
    Therefore, it holds with high probability that for large $\alpha$, $Pr[{\sum}_{u\in S_t}\omega_{t+1}(u) 
    \leq \alpha\cdot\frac{3}{4}] \geq \frac{7}{8}$.
    Note that, 
    we did not consider the effect when an agent turns to the \emph{quiet} state,
    which only makes the summation decrease and hence is not harmful. 
\end{IEEEproof}

\begin{lemma}
    \label{lemma8}
    There exists a constant $\hat{\alpha_2} > 1$,
    such that among $\gamma\log k$ slots (not necessarily consecutive)
    with ${\sum}_{u\in S_t}\omega_t(u) \geq \hat{\alpha_2}$ and sufficiently
    large $\gamma > 0$, there are at least $\frac{3}{4}\gamma\log k$ slots
    with ${\sum}_{u\in S_{t+1}}\omega_{t+1}(u) < \frac{3}{4}{\sum}_{u\in S_t}\omega_t(u)$,
    with probability $1-O(k^{-1})$.
\end{lemma}
\begin{IEEEproof}
    Let $T := \gamma\log k$, and $X_t$ be the random variable that indicates the value of 
    ${\sum}_{u\in S_{t+1}}\omega_{t+1}(u) / {\sum}_{u\in S_t}\omega_t(u)$. Then
    by Lemma~\ref{lemma7}, it holds that $Pr[X_t \leq \frac{3}{4}] \geq \frac{7}{8}$.
    Let $Y_t$ be the binary random variable that takes value $1$ if $X_t \leq \frac{3}{4}$.
    Note that given ${\sum}_{u\in S_t}\omega_t(u) \geq \hat{\alpha_2}$, $E[Y_t] \geq \frac{7}{8}$
    always hold. Hence, $E[\sum_{t=1}^{T}Y_t] \geq T\cdot\frac{7}{8}$, and it holds that
    $Pr[\sum_{t=1}^{T}Y_t\leq T\cdot\frac{3}{4}]$ by the Chernoff bound. That is with probability 
    $1-O(k^{-1})$, there are at least $T\cdot\frac{3}{4}$ slots $t$ with 
    ${\sum}_{u\in S_{t+1}}\omega_{t+1}(u) / {\sum}_{u\in S_t}\omega_t(u) \leq 3/4$, which completes the proof. 
\end{IEEEproof}


\begin{lemma}
    \label{lemma9}
    There exists a constant $\hat{\alpha_2} > 1$, such that
    during any period of $\gamma\log k$ slots with sufficiently large 
    $\gamma >0$, the probability that within the considered period there
    is a slot $t$ with ${\sum}_{u\in S_t}\omega_t(u) \leq \hat{\alpha_2}$
    is $1-O(k^{-1})$.
\end{lemma}
\begin{IEEEproof}
    Denote $T :=\gamma\log k$ and the period of $T$ starts from slot $t_0$.
    By Lemma~\ref{lemma8}, with probability at least $1-O{n^{-1}}$, it holds that 
    \begin{align*}
        {\sum}_{u\in S_{t_0+T}}\omega_{t_0+T}(u)  &\geq 
        {\sum}_{u\in S_{t_0}}\omega_{t_0}(u)\cdot(\frac{3}{4}\cdot\frac{3}{4}
        \cdot\frac{3}{4}\cdot 2)^{\frac{T}{4}} \\
       &={\sum}_{u\in S_{t_0}}\omega_{t_0}(u)\cdot(\frac{27}{32})^{\frac{T}{4}}.
    \end{align*} 
    Since $\sum_{u\in S_{t_0+T}}\omega_{t_0+T}(u) < k$ and $T = \gamma\log k$,
    we know that $\sum_{u\in S_{t_0+T}}\omega_{t_0+T}(u)$ is at most $\hat{\alpha_2}$ 
    for large enough $\gamma$.
\end{IEEEproof}


\begin{lemma}
    \label{lemma10}
    There exists a constant $\hat{\alpha} \geq 0.01$ such that
    for any time $t$ with ${\sum}_{u\in S_t}\omega_t(u)=\hat{\alpha}$,
    it holds that
    \begin{equation}
        Pr[{\sum}_{u\in S_{t+1}}\omega_{t+1}(u) \geq \hat{\alpha}\cdot \frac{4}{3}] \geq \frac{7}{8}.
    \end{equation}
\end{lemma}
\begin{IEEEproof}
    Denote X as the random variable that indicates the value of 
    ${\sum}_{u\in S_{t+1}}\omega_{t+1}(u)$. The probability that 
    there are no transmissions in time slot 
    $t$ is at least $4^{-\hat{\alpha}}$. Hence,
    \begin{equation*}
        Pr[X ={\sum}_{u\in S_t}2\cdot\omega_t(u)] 
        \geq 1 - 4^{-\hat{\alpha}}
    \end{equation*}
    which is at least $7/8$ when $\hat{\alpha}$ is close to $0.01$. Hence,
    \begin{equation*}
        X \geq \frac{7}{8}\cdot 2\cdot\hat{\alpha} + 
        \frac{1}{8}\cdot \frac{1}{2}\hat{\alpha} 
        > \frac{4}{3}\cdot\hat{\alpha}.
    \end{equation*}
    Therefore, it holds with high probability that for small $\hat{\alpha}$ close to $0.01$, $Pr[{\sum}_{u\in S_t}\omega_{t+1}(u) 
    \leq \hat{\alpha}\cdot\frac{4}{3}] \geq \frac{7}{8}$.


\end{IEEEproof}


\begin{lemma}
    \label{lemma11}
    There exists a constant $\hat{\alpha_1} > 0$, such that among $\gamma\log k$ 
    slots~(not necessarily consecutive) with ${\sum}_{u\in S_t}\omega_t(u)
    \leq \hat{\alpha_1}$ and sufficiently large $\gamma > 0$, there are at 
    least $\frac{3}{4}\gamma\log k$ slots with ${\sum}_{u\in S_{t+1}}\omega_{t+1}(u)
    \geq \frac{4}{3}{\sum}_{u\in S_t}\omega_t(u)$, with probability $1 - O(k^{-1})$.
\end{lemma}
\begin{IEEEproof}
    Denote $T := \gamma\log k$, $X_t$ as the random variable that indicates the value of 
    ${\sum}_{u\in S_{t+1}}\omega_{t+1}(u) / {\sum}_{u\in S_t}\omega_t(u)$, and 
    $Y_t$ to be the binary random variable that takes value $1$ if $X_t \leq \frac{4}{3}$.
    Note that given ${\sum}_{u\in S_t}\omega_t(u) \geq \hat{\alpha_1}$, $E[Y_t] \geq \frac{7}{8}$
    always hold. Hence, $E[\sum_{t=1}^{T}Y_t] \geq T\cdot\frac{7}{8}$, and it holds that
    $Pr[\sum_{t=1}^{T}Y_t\leq T\cdot\frac{3}{4}]$ by the Chernoff bound. That is with probability 
    $1-O(k^{-1})$, there are at least $T\cdot\frac{4}{3}$ slots $t$ with  
\end{IEEEproof}


\begin{lemma}
    \label{lemma12}
    Let $t_0$ be the first time slot in which ${\sum}_{u\in S_t}\omega_t(u)$
    drops below $\hat{\alpha_2}$. in the subsequent $T :=\tau\cdot\log k$ slots
    where $\tau > 0$ and $k$ is large enough, the following hold:
    \begin{itemize}
        \item[(1)] there are at least $\frac{3}{4}\cdot T$ slots $t$
        with ${\sum}_{u\in S_t}\omega_t(u) \leq \alpha_2$, where $\alpha_2 > \hat{\alpha_2}$
        is a constant.
        \item[(2)] there are at least $\frac{3}{4}\cdot T$ slots $t$
        with ${\sum}_{u\in S_t}\omega_t(u) \geq \alpha_1\cdot k$, where $\alpha_1 < \hat{\alpha_1}$
        is a constant.
    \end{itemize}
\end{lemma}
% \begin{IEEEproof}
%     First we prove the conclusion (1).

% \end{IEEEproof}

With Lemma~\ref{safe},~\ref{lemma7},~\ref{lemma8},~\ref{lemma9},~\ref{lemma10},~\ref{lemma11} and
\ref{lemma12}, we now prove Lemma~\ref{fast}.
% \begin{IEEEproof}
%     By Lemma~\ref{safe}, 
% \end{IEEEproof}


Then we show that when the number of agents not switching to 
the \emph{quiet} state is less than $\log k$, the process accelerates to 
only one agent not in \emph{quiet} state with high probability.
This is formulated as the following lemma.

\begin{lemma}
    \label{slow}
    At time $T_3 :=T_2 +\gamma_2 \cdot \log k = O(k)$ it holds with high
    probability that $|S_{T_3}| = 1$.
\end{lemma}
\begin{IEEEproof}
    After $T_2$ slots, it holds with high probability 
    that $|S_{T_2}| \leq \log k$. Then it takes at most $\gamma_2 \cdot \log k$
    slots to keep $\beta_1 \leq \sum_{u\in S_t} \omega_t^u \leq \beta_2$,
    where $\beta_1$ and $\beta_2$ are two constants.
    Afterwards, there is a time slot $T_3 :=T_2 +\gamma_2\cdot\log k$ such that
    $|S_{T_3}| = 1$, where $\gamma_2 > 0$ is a large enough constant. 
    Otherwise during the period from $T_2$ to $T_3$,
    with high probability there are 
    more than $\log k$ agents switching to
    the \emph{quiet} state.
\end{IEEEproof}

Finally we need to prove the following lemma.
\begin{lemma}
    \label{one}
    For time $T_3$ when there is only one agent not switching to \emph{quiet} state,
    at time $T_4 := 2 \cdot T_3 + \log k = O(k)$ it holds with high 
    probability that this agent successes to transmit.
\end{lemma}
\begin{IEEEproof}
    Since at time $T_3$ there is only one agent still attempting to transmit,
    the transmission probability will get to $\zeta$ at time $2\cdot T_3$ if it keeps
    listening. Note that if agent $u$ transmit with probability $\zeta$ for $\log k$ slots,
    then with high probability there exist one time slot in which agent $u$ successes to transmit.
\end{IEEEproof}

According to the Lemma~\ref{fast}, \ref{slow} and \ref{one},
we get the upper bound $O(k)$ of the {\pName} protocol with respect to
the clique size $k$. Note that, since the round of each agent may not be 
synchronized, {\pName} requires an overlap of at least $T_4$ time 
slots in a complete round between each agent.
This can be easily solved as we set $\hat{T} := 2\cdot T_4$ in Algorithm~\ref{CA}. 

\subsection{Dynamicity}
\label{boundPC}