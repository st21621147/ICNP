\section{Introduction}

Collecting detailed information about wild animals remains a significant technical challenge that
limits the ability of Ecologists to study them, the interactions between them, and the interactions 
between wild animals and their environment. One tools that has emerged a little over a decade ago
and has been gaining significance is {\em encounter detection and logging}.

Encounter detection systems consist of radio devices called {\em tags} that are attached to wild
animals (and sometimes also to fixed positions and to livestock). The radios transmit identification
packets periodically, usually once every few seconds. The radios also listen to such packets, not
necessarily continuously, and record data about received packets in persistent memory on the tag. The
radios are typically configured for short-range communication by using low transmit power and by 
using high data-rates (both limit the signal-to-noise ratio at the receiver). Because the radios
are configured for short-range communication, receiving a packet implies that the transmitting tag is
in close proximity to the receiving tag. Recent systems record with each packet a received signal strentgh
indication (RSSI), which helps to estimate the distance between the transmitter and receiver.
This is the main goal of these systems: to log close-proximity
events between two or more animals. The logs are downloaded either by physically retrieving the tags,
or remote download to base-stations placed in locations that the animals are known to frequent.

Such systems have gained popularity among Ecologists because the tags are relatively inexpensive and can
be very small, because deployment does not require much infrastructure in the field, and because close-proximity
encounters are key aspects of many significant events in the life of animals: mating, predation, transmission of infectious
diseases, and so on. We comment on some of these applications in Section~\ref{sec:related-work}.



 

