\begin{abstract}
Encounters between wild animals mark of the same or different species mark important events, such 
as mating, predation, and disease contraction, and they define the social structure of animal groups.
Several real-world systems designed to record such encounters have been implemented in the wildlife-research community.
In these systems, energy-restricted short-range radio devices are attached to wild animals, transmit identification
packets periodically, and receive and record packets from others.
%    Encounters are common interactions of wild animals.
%    Several real encounter-registration systems have been implemented
%    to help record the encounter process. In these systems, wild animals
%    are attached with energy-restricted radio devices called tags, 
%    to transmit packets periodically
%    and record the encounter process with others.
    However, non-trivial protocols for this problem have not been studies, and neither did the
    fundamental tradeoff 
    between minimizing the power consumption 
    and reducing the encounter latency. Our insight is that it is reasonable for a tag to
    increase the working frequency of its radio when encounter happens; otherwise it keeps
    the radio in a low-power mode.  

    In this paper, we propose a robust encounter registration protocol that with 
    high probability accomplishes the encounter registration 
    in $O(k)$ slots time, assuming $k$ is the number of encounter tags.
    Our protocol consists of two process: detect other tags with low energy consumption; 
    and make connections with the detected tags to record each other's ID. 
    We analyze the performance of this protocol under a formulated radio model and carry out a number of
    experiments to validate this model.
    To the best of our knowledge, this is the first practical protocol for a real 
    encounter-registration system.

\end{abstract}

