\begin{abstract}
Encounters between wild animals mark important events of 
the same or different species, such 
as mating, predation, and disease contraction, and they define 
the social structure of animal groups.
Several real-world systems designed to record such encounters 
have been implemented in the wildlife-research community.
In these systems, energy-restricted short-range radio devices 
are attached to wild animals, transmitting identification
packets periodically, receiving and recording packets from others.
%    Encounters are common interactions of wild animals.
%    Several real encounter-registration systems have been implemented
%    to help record the encounter process. In these systems, wild animals
%    are attached with energy-restricted radio devices called tags, 
%    to transmit packets periodically
%    and record the encounter process with others.
However, non-trivial protocols for this problem have not been studied, and neither did the
fundamental tradeoff 
between minimizing the power consumption 
and reducing the encounter latency. 
The key challenge is the uncertain mobility and herd characteristics of wild 
animals and our dominant strategy is to design a mechanism
to identify the existence of an encounter in real-time.
% Our insight is that it is reasonable for a tag to
% increase the working frequency of its radio when encounter happens; otherwise it keeps
% the radio in a low-power mode.  

In this paper, we propose a robust encounter registration protocol.
Our protocol consists of two processes: detect other tags with low energy consumption; 
and make connections with the detected tags to record each other's ID. 
We conduct experiments and extensive simulations to verify the scalability
of the protocol. Also, we present explicit animal models of mobility and evaluation results
show the strengths of our protocol in mobile wildlife environment.
To the best of our knowledge, this is the first practical 
protocol for a real encounter-registration system.

\end{abstract}

