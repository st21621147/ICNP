\begin{abstract}
    Encounter is a well-studied interaction between wild animals.
    Several real Encounter-registration systems have been implemented
    to help record the encounter process. In these systems, wild animals
    are attached with energy-restricted radio devices called tags, to transmit packets periodically
    and record the encounter process with others.
    However, these is no effective protocol for the encounter registration due to a paradox 
    between minimizing the power consumption of tags
    and reducing the encounter latency. Our insight is that it is reasonable a tag 
    increases the working frequency of its radio when encounter happens,and otherwise keeps
    the radio in a low-power mode. The key is to design mechanism
    to identify the existence of encounter in real-time. 



    In this paper, we propose an adaptive encounter protocol that with 
    high probability  accomplishes the encounter registration 
    in $O(k)$ slots time, assuming $k$ is the number of encounter tags.
    Our protocol consists of two process: detect other tags with low energy consumption; 
    and make connection with the detected tags to identify each other and record the ID. 
    We analyze the performance of this protocol under a formulated radio model and carry out a number of
    experiments to validate this radio model.
    To the best of our knowledge, this is the first practical protocol for a real 
    Encounter-registration system.

\end{abstract}

