\section{System Model}
\label{sectionmodel}


\subsection{Communication Model}



An agent that has its radio on can choose to be in the $Transmit$ state
or the $Listen$ state:
\begin{itemize}
\item \textbf{Transmit state:} an agent transmits (broadcasts) 
a packet containing its ID on the channel;
\item  \textbf{Listen state:} an agent listens on 
the channel to receive messages from peers.
\end{itemize}

Suppose time is divided into slots of equal length $t_0$
, which is assumed to be sufficient to finish a complete
communication process (one agent transmits a message including its ID and
a peer receives the message).

\textbf{Duty cycle mechanism.} In the wildlife tracing system \emph{ATLAS}, 
each agent (animal) is equipped with energy-restricted tags.
An agent has to turn off the radio to save
energy for most of the time and may only be active 
(transmitting or receiving) during a fraction $\theta$ of the time.
%, where $\theta$ is typically $1/100$ or less.
In each time slot, an agent $u_i$ adopt an action as:
$$ s_i^t=\left\{
\begin{aligned}
&Sleep  & & & &{sleep~ with~ probability~ (1-\theta_i)}  	 \\
&Transmit  & & & &{transmit~ with~ probability~ \theta_i p}	\\
&Listen  & & & &{listen~ with~ probability~\theta_i(1-p)}	\\
\end{aligned}
\right.
$$
\emph{Duty cycle} is defined as the fraction of time an agent turns its radio on, 
which is formulated as:
$$\theta_i=\frac{|\{t: 0\leq t<t_0, s_i(t) \in \{Transimit,Listen\}\}|}{t_0}.
$$

\subsection{Interference Model}

We consider graph model as the interference model in our protocol. 
Graph model is a popular one that enables the development of
efficient algorithms for crucial networking problems. Some other models,
such as the signal to interference plus noise ratio (\emph{SINR}) model,
are more complicated and lack good algorithmic features. 
In addition, it is shown that SINR can be transformed to the graph model by
particular means.

In the graph model, we assume that the communication range of each agent 
is $D$ meters and an agent $u_i$ can discover its nearby peer $u_j$ 
in time slot $t$ if and only if $u_j$ is the only peer in $u_i$'s range 
that transmits and $u_i$ is listening in the slot.

Suppose the distance between two nodes $u_i$ and $u_j$ is $\Delta_{i,j}$.
The connectivity of the edge $e_{i,j}$ in the graph model is formulated as:
$$ \omega_e=\left\{
\begin{aligned}
&0  & & &   &D \leq \Delta_{i,j}	 \\
%&\phi_{i,j}  & & & 	&d \leq \Delta_{i,j} \leq D\\
&1  & & & &\Delta_{i,j} \leq D	\\
\end{aligned}
\right.
$$
%where $\varphi(\Delta)$ is an inverse function 
%(e.g., the weight is inversely proportional to the distance).
 
An agent $u_i$ can discover its peer $u_j$ in
time slot $t$ if and only if $u_j$ is the only peer transmitting 
in the range of $u_i$ ($\Delta_{i,j} \leq D$) and $u_i$ is listening in the slot.
An agent suffers from collisions when it receives simultaneous messages. 


\textbf{Collision detection mechanism.}
A listening agent can distinguish whether collisions occur or no nearby peer 
is transmitting, apart from successful discovery. 

\subsection{Problem formulation}

In this paper, we study the encounter problem in a wildlife 
tracing system. We first give the definition of the \emph{Encounter process}.
\begin{definition}
Encounter is defined as the process that 
an agent detects and monitors a period of close proximity a wildlife tracking system
to another peer. 
\end{definition}

The problem in this paper is formulated as follows:
\begin{problem}
    We define a $D$-encounter problem as a least
    time interval of $T_0$ during which agent $u_i$ and $u_j$ satisfying
    $\Delta_{i,j}(t) \leq D$.
\end{problem}

The aim is to design a protocol to guarantee any $D$-encounter 
can be detected and recorded within $T$ time slots with high probability 
($T$ is the upper bound of the protocol, e.g. $T=O(n\ln n)$).

We look into the problem and find the key challenge is the uncertainty of dynamic movements
of agents. Despite the dynamicity in this real system, 
when $T$ is relatively short (compared to the kinematic velocity of agents) in the 
real world (e.g., less than $1$ second),
we assume the communication connectivity of the agents is constant in $T$
time slots. 

\#Clique description (TBD) ...
