\section{System Model}
\label{sectionmodel}

In this paper, we study the encounter problem in a wildlife 
tracing system. We call individual animals as \emph{agents}, 
and \emph{peers} are referred as other agents that distinguish from 
a specific one.
The definition of the \emph{encounter process} is formulated as follows.
\begin{definition}
Encounter is defined as the process that 
an agent detects and records other peer(s) if they keep a period of 
close proximity $\Delta \leq D$
in the wildlife tracking system. 
\end{definition}

In the following, we describe the system model for theoretical analysis in this paper.



\subsection{Communication Model}

% First, we present the communication model in this paper.

In the  wildlife tracing system {\sysname}, the encounter behavior  
is a common biological phenomenon and
happens when more than one agents gather closely, constituting a 
single clique of size $k$~($k \geq 2$).
Note that, $k$ is not known to each agent and the whole 
clique composes a sing-hop network for communication due to the proximity. 

Each agent is equipped with a radio tag. 
An agent that has its radio on can choose to be in the $transmit$ state
or the $listen$ state:
\begin{itemize}
\item \textbf{Transmit state:} an agent transmits (broadcasts) 
a message containing its ID on the channel;
\item  \textbf{Listen state:} an agent listens on 
the channel to receive messages from peers.
\end{itemize}
We also call an agent keeps in the \emph{listen} state for a period of consecutive slots 
as \emph{quiet} state.

Suppose time is divided into synchronized slots of equal 
length $2\hat{t_0}$~\cite{Xu2005Lightweight, Sivrikaya2004Time}
, where $\hat{t_0}$ is assumed to be sufficient to finish a complete
communication process (one agent transmits a message including its ID and
a peer receives the message).

An agent transmits successfully in a time slot if and only if 
it is the only one transmitting and all the other peer(s) will
receive its message and record its ID in this single-hop network. Otherwise the channel is detected
as \emph{idle} if there is no transmission and \emph{busy} if there 
are simultaneous messages incurring collisions on the channel.

In the wildlife tracing system {\sysname}, 
on the one hand,
each agent is equipped with an energy-restricted tag;
on the other hand, encounter process happens occasionally, and thus
it is a waste of battery energy if an agent turns on the radio while it does 
not encounter with any peer(s) at the moment. 
Therefore, in order to keep a balance between the energy consumption 
and the efficiency of the encounter process, we introduce the duty cycle mechanism~\cite{Zhang2017Performance}.

\textbf{Duty cycle mechanism.} 
An agent has the capability to turn off the radio to save
energy for most of the time, and may only be active 
(transmitting or receiving) during a fraction $\theta$ of the time.
%, where $\theta$ is typically $1/100$ or less.

Incorporating the duty cycle mechanism into the Mac layer of the radio tag, 
in each time slot an agent $u_i$ is able to adopt an action as:
$$ s_i^t=\left\{
\begin{aligned}
&Sleep  & & & &{sleep~ with~ probability~ (1-\theta_i)}  	 \\
&Transmit  & & & &{transmit~ with~ probability~ \theta_i p}	\\
&Listen  & & & &{listen~ with~ probability~\theta_i(1-p)}	\\
\end{aligned}
\right.
$$
\emph{Duty cycle} is defined as the fraction of time an agent turns its radio on, 
which is formulated as:
$$\theta_i=\frac{|\{t: 0\leq t<t_0, s_i(t) \in \{Transimit,Listen\}\}|}{t_0}.
$$

% \subsection{Interference Model}

% In reality, wireless transmission suffers from interference.
% We consider graph model as the interference model in our protocol. 
% Graph model is a popular one that enables the development of
% efficient algorithms for crucial networking problems. Some other models,
% such as the signal to interference plus noise ratio (\emph{SINR}) model,
% are more complicated and lack good algorithmic features. 
% In addition, it is shown that SINR can be transformed to the graph model by
% particular means.

% In the graph model, we assume that the communication range of each agent 
% is $D$ meters and an agent $u_i$ can discover its nearby peer $u_j$ 
% in time slot $t$ if and only if $u_j$ is the only peer in $u_i$'s range 
% that transmits and $u_i$ is listening in the slot.




% Suppose the distance between two nodes $u_i$ and $u_j$ is $\Delta_{i,j}$.
% The connectivity of the edge $e_{i,j}$ in the graph model is formulated as:
% $$ \omega_e=\left\{
% \begin{aligned}
% &0  & & &   &D \leq \Delta_{i,j}	 \\
% %&\phi_{i,j}  & & & 	&d \leq \Delta_{i,j} \leq D\\
% &1  & & & &\Delta_{i,j} \leq D	\\
% \end{aligned}
% \right.
% $$
% %where $\varphi(\Delta)$ is an inverse function 
% %(e.g., the weight is inversely proportional to the distance).
 
% An agent $u_i$ can discover its peer $u_j$ in
% time slot $t$ if and only if $u_j$ is the only peer transmitting 
% in the range of $u_i$ ($\Delta_{i,j} \leq D$) and $u_i$ is listening in the slot.
% An agent suffers from collisions when it receives simultaneous messages. 

Next, we introduce another efficient technique called collision detection mechanism.
This technique is carried out by the physical carrier sensing~\cite{Yang2005On}, which is part of 
the $802.11$ standard, and provided by a Clear Channel Assessment (CCS) circuit.

\textbf{Collision detection mechanism.}
A listening agent can distinguish whether the channel is \emph{idle} or \emph{busy}, 
apart from successfully receiving a message. 

\subsection{Problem formulation}

We formulate the problem in this paper as follows.
\begin{problem}
    Consider $\hat{T}$ slots which is a small enough period in reality.
    We define an encounter problem as to design a protocol to guarantee 
    all the agents in the clique can receive message from each other at least once 
    if they encounter for at least $\hat{T}$ time slots and record the encounter process. 
\end{problem}


We look into the problem and find the key challenge is 
the uncertainty of dynamic movements
of agents. Despite the dynamicity in this real system, 
when $\hat{T}$ is short enough relative to 
the time required for an agent to move a step~(a short distance)
in reality~(e.g., less than $1$ second),
we can make a reasonable assumption that the communication 
connectivity of the agents is stable during each $\hat{T}$ time slots. 

